\documentclass [14pt]{beamer}

\usepackage{relsize}
\usepackage[absolute,overlay]{textpos} 
\usepackage[colorgrid,texcoord]{eso-pic} 
\usepackage{tikz}

\tikzset{
  every overlay node/.style={
    draw=black,fill=white,rounded corners,anchor=north west,
  },
}
% Usage:
% \tikzoverlay at (-1cm,-5cm) {content};
% or
% \tikzoverlay[text width=5cm] at (-1cm,-5cm) {content};
\def\tikzoverlay{%
   \tikz[baseline,overlay]\node[every overlay node]
}%

%\DeclareMathOperator{\diag}{diag}

\mode<presentation>
{
%	\usetheme{Boadilla}
%	\usecolortheme{beaver}
  \usetheme{default}
}

\setbeamertemplate{background}[grid][step=5mm]
\setlength{\TPHorizModule}{5mm}
\setlength{\TPVertModule}{5mm}

\title[Turnkey reproducibility]
{Turnkey reproducibility}

\author[G. Oxberry]
{\textbf{Geoffrey M. Oxberry}}

\institute[LLNL]
{
 Lawrence Livermore National Laboratory \\
Computational Engineering Division \\
Energy Conversion and Storage
}

\date[ICERM 2012]
{December 13, 2012}

\begin{document}

% Title slide
\begin{frame}
\titlepage
\end{frame}

% Establish need
\begin{frame}
\frametitle{Reproducing someone's work can be hard}
\begin{itemize}
\item Need to install necessary software (assume open source)
\begin{itemize}
\item Takes time, expertise, patience
\item Could affect system stability
\end{itemize}
\item Could wrap source in a VM
\begin{itemize}
\item Requires lots of storage (usually $\geq 300$ MB)
\item No separation of source code \& environment
\end{itemize}
\item High barrier means \textbf{people don't run the code}
\end{itemize}
\end{frame}

% Establish scope of problem
\begin{frame}
\frametitle{Need to lower barrier to reproducibility}
\begin{itemize}
\item Separate source code and environment
\item Source code: store in Git repo
\item Environment options:
\begin{itemize}
\item Supply separate VM (lots of storage)
\item Use external service (like RunMyCode)
\item Dynamically spawn VM from configuration files
\end{itemize}
\end{itemize}
\end{frame}

% Propose solution
\begin{frame}
\frametitle{Dynamically spawning VMs\\ provides most flexibility}
\begin{itemize}
\item Configuration stored in small text files
\item Reduces storage needed to specify VM
\item Can pull in source from Git as part of configuration
\item Easy to spawn/destroy (one command) locally
\item Only depends on reputable open source software
\item No language limits; can adapt to cloud, clusters
\end{itemize}
\end{frame}

% Establish details
\begin{frame}
\frametitle{Requires four open-source packages}
\begin{itemize}
\item (\textbf{Note:} Replace with graphics after this draft.)
\item Vagrant
\item Puppet (or Chef, or shell scripts)
\item VirtualBox
\item Git
\end{itemize}
\end{frame}

\begin{frame}
\frametitle{Vagrant spawns the VMs}
\begin{itemize}
\item http://www.vagrantup.com/
\item Download minimal machine image
\begin{itemize}
\item List of images on external sites: http://www.vagrantbox.es/
\item Create an image with VeeWee:
  http://www.github.com/jedi4ever/veewee
\item Create an image using instructions in Vagrant docs
\end{itemize}
\item Configure VM via text files, using:
\begin{itemize}
\item Puppet
\item Chef
\item Shell script  
\end{itemize}
\item Vagrantfile (like Makefile) specifies machine image, location of config files
\end{itemize}
\end{frame}

\begin{frame}
\frametitle{Puppet (Chef, shell script)\\
configures VMs}
\begin{itemize}
\item Puppet: http://www.puppetlabs.com/
\item Chef: http://wiki.opscode.com/display/chef/home
\item Text file specifying configuration
\begin{itemize}
\item Either shell script or domain specific language (DSL)
\item DSL-based files easy to read
\item Small, can version in Git
\item Can pull in software using Git, RubyGems, Pip, etc.
\end{itemize}
\end{itemize}
\end{frame}

\begin{frame}
\frametitle{VirtualBox runs VMs}
\begin{itemize}
\item http://www.virtualbox.org/
\item Supported by Oracle
\item Cross-platform: Linux, Windows, OS X
\item For large parallel jobs, replace with:
\begin{itemize}
\item Amazon Web Services' CloudFormation
\item StarCluster
\end{itemize}
\end{itemize}
\end{frame}

\begin{frame}
\frametitle{Example: Update matrix exponential in SciPy}
\begin{itemize}
\item Source Git repo: \textbf{(add GitHub URL here)}
\item Environment Git repo: \textbf{(add GitHub URL here)}
\begin{itemize}
\item Vagrantfile to specify box to spawn
\item Configuration in Puppet
\item README with rections for running software
\end{itemize}
\item Run unit tests for reproducibility
\end{itemize}
\end{frame}

% Acknowledgments slide
\begin{frame}
\frametitle{Acknowledgments}
\begin{itemize}
\item Dr.\ Matt McNenly
\item Dr.\ Dan Flowers
\item Dr.\ David I.\ Ketcheson
\item Dr.\ Aron Ahmadia
\item US Department of Energy
\end{itemize}
\end{frame}

\begin{frame}
\end{frame}

\end{document}