\documentclass [14pt]{beamer}

\usepackage{relsize}
\usepackage[absolute,overlay]{textpos} 
\usepackage[colorgrid,texcoord]{eso-pic} 
\usepackage{hyperref}

% Taken from
% http://tex.stackexchange.com/questions/13423/how-to-color-href-links-in-beamer
\definecolor{links}{HTML}{660099}
\hypersetup{colorlinks,linkcolor=,urlcolor=links}

\mode<presentation>
{
	\usetheme{Boadilla}
%	\usecolortheme{beaver}
%  \usetheme{default}
  \usecolortheme{default}
}

%\setbeamertemplate{background}[grid][step=5mm]
\setlength{\TPHorizModule}{5mm}
\setlength{\TPVertModule}{5mm}

\title[Turnkey reproducibility]
{Towards turnkey reproducibility}

\subtitle[G. Oxberry]
{\textbf{Geoffrey M. Oxberry}}

\author[]{\small{Lawrence Livermore National Laboratory \\
Computational Engineering Division \\
Energy Conversion and Storage}}

\institute[LLNL-PRES-607580-DRAFT]
{
\footnotesize{This work performed under the auspices of the U.S. Department of Energy by Lawrence Livermore National Laboratory under Contract DE-AC52-07NA27344}.
}

\date[ICERM 2012]
{December 13, 2012}

\begin{document}

% Title slide
\begin{frame}
\titlepage
\end{frame}

% Establish need
\begin{frame}
\frametitle{Problem: Reproducing someone's work can be hard}
\begin{itemize}
\item Need to install necessary software (assume open source)
\begin{itemize}
\item Takes time, expertise, patience, privileges
\item Could affect system stability
\end{itemize}
\item Could wrap source in VM (virtual machine) image
\begin{itemize}
\item Usually requires $\geq 300$ MB to host; big, unwieldy
\item No separation of source code \& environment means no flexibility
\end{itemize}
\item High barrier means \textbf{people don't run the code}
\item \textbf{Lower hosting \& time barrier} by specifying environment in
  separate repo using configuration management software
\end{itemize}
\end{frame}

% ICERM wants a three-slide version; I'll give them three slides,
% NOT counting the title and acknowledgments slides.

% Propose solution:
\begin{frame}
\frametitle{Solution: Specify environment with configuration management software}
\begin{itemize}
\item Config management tools specify config in text
  files
\begin{itemize}
\item Shell scripts (simplest, fewest prepackaged features)
\item Puppet (\url{puppetlabs.com})
\item Chef (\url{wiki.opscode.com})
\item Fabric (\url{http://docs.fabfile.org/en/1.5/})
\item Blueprint (\url{http://devstructure.com/blueprint/}
%\item Hashdist (\url{http://hashdist.readthedocs.org/en/latest/})
\end{itemize}
\item Instantiate config using virtualization tools
\begin{itemize}
\item Serial, small parallel jobs: Vagrant (\url{vagrantup.com}) + VirtualBox
  (\url{virtualbox.org})
\item StarCluster (\url{star.mit.edu/cluster})
\item CloudFormation (\url{aws.amazon.com/cloudformation})
%\item PiCloud (\url{http://www.picloud.com/})
\item Any other virtualization software + hardware
\item Use web services instead (like Wakari, RunMyCode)
\end{itemize}
\item Idea is \textbf{flexibility}: pick \& choose (even none, mix)
\end{itemize}
\end{frame}

% Example slide:
\begin{frame}
\frametitle{Example: Install Python interface for DASSL}
\begin{itemize}
\item DASSL: differential-algebraic equation solver package in Fortran
  (L. Petzold)
\item PyDAS: Python interface to DASSL (J. W. Allen, on GitHub)
\item Example: Solve Robertson problem in IPython notebook using PyDAS
\item Presentation, environment and source repos on
  \url{https://github.com/goxberry} (all labeled with ICERM-2012)
\begin{itemize}
\item Requires Vagrant + VirtualBox
\item Vagrantfile to specify VM to create (here, Ubuntu 12.04)
\item Configuration in Puppet
\item README with directions for running software
\end{itemize}
\end{itemize}
\end{frame}

% Acknowledgments slide
\begin{frame}
\frametitle{Acknowledgments}
\begin{itemize}
\item Dr.\ Matt McNenly
\item Dr.\ Dan Flowers
\item Dr.\ David I.\ Ketcheson
\item Dr.\ Aron Ahmadia
\item US DOE
\item DOE CSGF
\item Gurpreet Singh, program manager for the DOE EERE Advanced
  Combustion Engine Program, for his continued support of the Advanced
  Combustion Numerics project at LLNL
\end{itemize}
\end{frame}

\begin{frame}
\end{frame}

\end{document}