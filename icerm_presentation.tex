\documentclass [14pt]{beamer}

\usepackage{relsize}
\usepackage[absolute,overlay]{textpos} 
\usepackage[colorgrid,texcoord]{eso-pic} 
\usepackage{hyperref}

% Taken from
% http://tex.stackexchange.com/questions/13423/how-to-color-href-links-in-beamer
\definecolor{links}{HTML}{660099}
\hypersetup{colorlinks,linkcolor=,urlcolor=links}

\mode<presentation>
{
	\usetheme{Boadilla}
%	\usecolortheme{beaver}
%  \usetheme{default}
  \usecolortheme{default}
}

%\setbeamertemplate{background}[grid][step=5mm]
\setlength{\TPHorizModule}{5mm}
\setlength{\TPVertModule}{5mm}

\title[Turnkey reproducibility]
{Towards turnkey reproducibility}

\subtitle[G. Oxberry]
{\textbf{Geoffrey M. Oxberry}}

\author[]{\small{Lawrence Livermore National Laboratory \\
Computational Engineering Division \\
Energy Conversion and Storage}}

\institute[LLNL-PRES-XXXXXX]
{
\footnotesize{This work performed under the auspices of the U.S. Department of Energy by Lawrence Livermore National Laboratory under Contract DE-AC52-07NA27344}.
}

\date[ICERM 2012]
{December 13, 2012}

\begin{document}

% Title slide
\begin{frame}
\titlepage
\end{frame}

% Establish need
\begin{frame}
\frametitle{Reproducing someone's work can be hard}
\begin{itemize}
\item Need to install necessary software (assume open source)
\begin{itemize}
\item Takes time, expertise, patience, privileges
\item Could affect system stability
\end{itemize}
\item Could wrap source in VM (virtual machine) image
\begin{itemize}
\item Usually requires $\geq 300$ MB to host; big, unwieldy
\item No separation of source code \& environment means no flexibility
\end{itemize}
\item High barrier means \textbf{people don't run the code}
\item \textbf{Lower hosting \& time barrier} by specifying environment in
  separate repo using configuration management software
\end{itemize}
\end{frame}

% ICERM wants a three-slide version; I'll give them three slides,
% NOT counting the title and acknowledgments slides.

%Propose solution:
\begin{frame}
\frametitle{Specify environment with configuration management software}
\begin{itemize}
\item Config management tools specify config in text
  files
\begin{itemize}
\item Shell scripts (simplest, fewest prepackaged features)
\item Puppet (\url{puppetlabs.com})
\item Chef (\url{wiki.opscode.com})
\end{itemize}
\item Instantiate config using virtualization tools
\begin{itemize}
\item Serial, small parallel jobs: Vagrant (\url{vagrantup.com}) + VirtualBox
  (\url{virtualbox.org})
\item StarCluster (\url{star.mit.edu/cluster})
\item CloudFormation (\url{aws.amazon.com/cloudformation})
\item Any other virtualization software
\item Web services? (Maybe RunMyCode?)
\end{itemize}
\item Can instantiate config in host OS, if desired
\end{itemize}
\end{frame}

\begin{frame}
\frametitle{Example: Add matrix 1-norm estimation to SciPy}
\begin{itemize}
\item Needed for Higham's 2009, 2010 matrix exponential algorithms
\item Presentation repo: \textbf{(add URL here)}
\item Source repo: \textbf{(add URL here)}
\item Environment repo: \textbf{(add URL here)}
\begin{itemize}
\item Requires Vagrant + VirtualBox
\item Vagrantfile to specify box to spawn
\item Configuration in Puppet
\item README with directions for running software
\end{itemize}
\item Unit tests included to demonstrate reproducibility
\end{itemize}
\end{frame}

% Establish scope of problem
%\begin{frame}
%\frametitle{Need to lower barrier to reproducibility}
%\begin{itemize}
%\item Separate source code and environment
%\item Source code: store in Git repo
%\item Environment options:
%\begin{itemize}
%\item Supply separate VM (big binary file to host)
%\item Use external service (like RunMyCode)
%\item Dynamically spawn VM from configuration files
%\end{itemize}
%\item Specifying environment in text files yields maximum flexibility!
%\end{itemize}
%\end{frame}

% Propose solution
%\begin{frame}
%\frametitle{Dynamically spawning VMs\\ provides most flexibility}
%\begin{itemize}
%\item Configuration stored in small text files
%\item Reduces storage needed to specify VM
%\item Can pull in source from Git as part of configuration
%\item Easy to spawn/destroy (one command) locally
%\item Only depends on reputable open source software
%\item No language limits; can adapt to cloud, clusters
%\end{itemize}
%\end{frame}

% Establish details
%\begin{frame}
%\frametitle{Requires four open-source packages}
%\begin{itemize}
%\item (\textbf{Note:} Replace with graphics after this draft.)
%\item Vagrant
%\item Puppet (or Chef, or shell scripts)
%\item VirtualBox
%\item Git
%\end{itemize}
%\end{frame}

%\begin{frame}
%\frametitle{Vagrant spawns the VMs}
%\begin{itemize}
%\item \url{http://www.vagrantup.com/}
%\item Download minimal machine image from:
%\begin{itemize}
%\item List of images on external sites: \url{http://www.vagrantbox.es/}
%\item Image created with VeeWee:
%  \url{http://www.github.com/jedi4ever/veewee}
%\item Image created using instructions in Vagrant docs
%\end{itemize}
%\item Configure VM via text files, using:
%\begin{itemize}
%\item Puppet
%\item Chef
%\item Shell script  
%\end{itemize}
%\item Vagrantfile (like Makefile) specifies machine image, location of config files
%\end{itemize}
%\end{frame}

%\begin{frame}
%\frametitle{Puppet (Chef, shell script)\\
%configures VMs}
%\begin{itemize}
%\item Puppet: \url{http://www.puppetlabs.com/}
%\item Chef: \url{http://wiki.opscode.com/display/chef/home}
%\item Text file specifying configuration
%\begin{itemize}
%\item Either shell script or domain specific language (DSL)
%\item DSL-based files easy to read
%\item Small, can version in Git
%\item Can pull in software using package managers, Git, RubyGems, Pip, etc.
%\end{itemize}
%\end{itemize}
%\end{frame}

%\begin{frame}
%\frametitle{VirtualBox runs VMs}
%\begin{itemize}
%\item \url{http://www.virtualbox.org/}
%\item Supported by Oracle
%\item Cross-platform: Linux, Windows, OS X
%\item Hosts VM locally (like VMware)
%\item Can replace Vagrant \& VirtualBox with:
%\begin{itemize}
%\item Amazon Web Services' CloudFormation (parallel jobs)
%\item StarCluster (parallel jobs)
%\end{itemize}
%\end{itemize}
%\end{frame}

%\begin{frame}
%\frametitle{Example: Update matrix exponential in SciPy}
%\begin{itemize}
%\item Source Git repo: \textbf{(add GitHub URL here)}
%\item Environment Git repo: \textbf{(add GitHub URL here)}
%\begin{itemize}
%\item Vagrantfile to specify box to spawn
%\item Configuration in Puppet
%\item README with rections for running software
%\end{itemize}
%\item Run unit tests for reproducibility
%\end{itemize}
%\end{frame}

% Acknowledgments slide
\begin{frame}
\frametitle{Acknowledgments}
\begin{itemize}
\item Dr.\ Matt McNenly
\item Dr.\ Dan Flowers
\item Dr.\ David I.\ Ketcheson
\item Dr.\ Aron Ahmadia
\item US DOE
\item DOE CSGF
\item Gurpreet Singh, program manager for the DOE EERE Advanced
  Combustion Engine Program, for his continued support of the Advanced
  Combustion Numerics project at LLNL
\end{itemize}
\end{frame}

\begin{frame}
\end{frame}

\end{document}